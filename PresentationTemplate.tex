%##############################################################################
%  PRESENTATION TEMPLATE.TEX
%
%  DESCRIPTION
%    This document is a template TEX file for a BEAMER presentation.
%
%  AUTHOR
%    Taylor Hines, taylor.hines@asu.edu
%
%  MODIFICATION HISTORY
%    05/11/09 - Date of creation
%
%  NOTES
%    You definitely do not need to acknowledge me in any way if you
%    use this template.  This document has probably been borderline
%    copied from other people.  In fact, you should probably just
%    delete this paragraph.
%
%##############################################################################
%##############################################################################
%
%       TITLE:
%
%       AUTHOR:
%
%       DATE:
%
%
%##############################################################################
%##############################################################################
%
%   SETTINGS AND PACKAGES
%
\documentclass[compress]{beamer}

\input{BEAMERoptions.tex}

\usepackage{graphics,multimedia,times,url}
\usepackage[english]{babel}

\title{Sample BEAMER Presentation}
\author{ Matt Kinsinger\thanks{mkinsing@asu.edu} }
\date{May 9, 2017}

\institute[Mathematics and Statistics]{
\includegraphics[height=.85cm]{ASUlogo.pdf} \\
{\color{ASUred} SCHOOL OF \textbf{MATHEMATICAL AND STATISTICAL SCIENCES}}}

%#############################################################################
%#############################################################################
%
%   BEGIN PRESENTATION
%
\begin{document}

%#############################################################################
%#############################################################################
%
%       TITLE FRAME
%
\begin{frame}[plain]
	\titlepage
	\transboxout
\end{frame}

%#############################################################################
%#############################################################################
%
%       INTRODUCTION
%
\section{Introduction}\label{Introduction}

\begin{frame}\frametitle{Introduction} %a frame is a slide
\begin{itemize}
\item Heat as a function of 1-dimensional space and time: $u(x,t)$
\item Notation for $u$ evaluated at discretized points in space and time $$u(x_i,t_n)=u_i^n$$
\item The heat equation, a second order partial differential equation
$$\frac{\partial u}{\partial t}=k\frac{\partial ^2u}{\partial x^2 }$$
\item Euler's method approximation
$$u_i^{n+1}\approx u_i^n+\frac{k\Delta t}{\Delta x^2}\left[u_{i+1}^n-2u_i^n+u_{i-1}^n\right]$$
\end{itemize}
\end{frame}

\begin{frame}
\begin{itemize}
\item We need the initial temperature profile of the rod and the temperatures for all points in time of the endpoints of the rod
\begin{itemize}
\item We will give these values in our MatLab code
\end{itemize}
\item Example 1
\begin{itemize}
\item Length of rod, L:		1 meter 
\item Number of points:		50 
\item Total time, $t_f$:	5 seconds 
\item Number of time points:	2000 
\item Diffusivity constant:		k
\item $r=k\frac{\Delta t}{\Delta x^2}$ \hspace{.5in}$r<0.50$ required
\item $u(1,:)=\sin \left(\frac{2\pi}{t_f}\;t\right)$ \hspace{.5in} Left endpoint boundary conditions
\item $u(N,:)=\cos \left(\frac{2\pi}{t_f}\;t\right)$ \hspace{.5in}Right endpoint boundary conditions
\item $u(:,1)=\sin \left(\frac{2\pi}{L}\right)\;x$ \hspace{.5in}Initial conditions
\end{itemize}
\end{itemize}
\end{frame}

\begin{frame}
\[
\begin{array}{c c}
\includegraphics[scale=.5]{fig1} 	&	\includegraphics[scale=.5]{fig2}	
\end{array}	
\]
\end{frame}

\begin{frame}
\[
\begin{array}{c c}
\includegraphics[scale=.5]{fig3}	&\includegraphics[scale=.5]{fig4}	
\end{array}	
\]
\end{frame} 

 	

\begin{frame}
\begin{itemize}
\item A better approximation method: Second order Runge-Katta 
\begin{itemize}
\item Approximate the slopes at the endpoints of each time interval
\begin{itemize}
\item $\widetilde{K}_{i+1}$ is found using an approximation for $u_i^{n+1}$
\end{itemize}
\item Use the average of these slopes to linearly approximate the solution over the interval
\end{itemize}

$$u_i^{n+1}=u_i^n+\Delta t\left[\frac{K_{i}+\widetilde{K}_{i+1} }{2}\right]$$
\item $K_{i}\approx \text{slope at $u_i^n$}$
\item $\widetilde{K}_{i+1}\approx \text{slope at $u_i^{n+1}$}$
\end{itemize}
\end{frame}

\begin{frame}
\begin{itemize}
\item Pieces to the approximation
\vspace{.2in}

\begin{itemize}
\item $K_{i}=\frac{u_i^{n+1}-u_i^n}{\Delta t}=\frac{k}{\Delta x^2}\left[u_{i-1}^n-2u_i^n+u_{i+1}^n\right]$
\vspace{.2in}

\item $\tilde{u}_i^{n+1}=u_i^{n}+\frac{k\Delta t}{\Delta x^2}\left[u_{i-1}^n-2u_i^n+u_{i+1}^n\right]$
\vspace{.2in}

\item $\widetilde{K}_{i+1}=\frac{u_{i}^{n+2}-u_i^{n+1}}{\Delta t}=\frac{k}{\Delta x^2}\left[\tilde{u}_{i-1}^{n+1}-2\tilde{u}_i^{n+1}+\tilde{u}_{i+1}^{n+1}\right]$
\end{itemize}
\end{itemize}
\end{frame}

\begin{frame}
\begin{itemize}
\item Example 2
\begin{itemize}
\item All parameters the same as in example 1
\item Still had the same limiting diffusivity constant, $k\approx 0.083$ 
\end{itemize}
\end{itemize}
\[
\begin{array}{c c}
\includegraphics[scale=.5]{imp_euler_fig1} 	&	\includegraphics[scale=.5]{imp_euler_fig2} 	
\end{array}	
\]
\end{frame}

\begin{frame}
\[
\begin{array}{c c}
\includegraphics[scale=.5]{imp_euler_fig2} 	&	\includegraphics[scale=.5]{imp_euler_fig4} 	
\end{array}	
\]
\end{frame}


\begin{frame}
\begin{center}
Implicit Time Stepping
\end{center}
\begin{itemize}
\item Benefits
\begin{itemize}
\item Increased stabilty of approximation
\item Able to handle larger diffusivity constants
\end{itemize}
\item Drawbacks
\begin{itemize}
\item Need to solve a linear system  $Ax=b$
\end{itemize}
\end{itemize}
\end{frame}

\begin{frame}
\begin{itemize}
\item Setting up the approximation
\begin{itemize}
\item Use the Taylor's series expansion, but the RHS of our final approximation expression is in terms of $u_x^{t_{n+1}}$ rather than $u_x^{t_{n}}$. 
\begin{align*}
\frac{u_i^{n+1}-u_i^n}{\Delta t}	&=\frac{k}{\Delta x^2}\Big[u_{i-1}^{n+1}-2u_i^{n+1}+u_{i+1}^{n+1}\Big]	\\
									&																			\\
u_i^n								&=u_i^{n+1}-\frac{k\Delta t}{\Delta x^2}\Big[u_{i-1}^{n+1}-2u_i^{n+1}+u_{i+1}^{n+1}\Big] 	\\
									&																			\\
u_i^n								&=u_i^{n+1}\Big[1+2r\Big]-r\Big[u_{i-1}^{n+1}+u_{i+1}^{n+1}\Big]			&&,r=\frac{k\Delta t}{\Delta x^2}						
\end{align*}
\item We are using the value at the $next$ time step
\end{itemize}
\end{itemize}
\end{frame}

\begin{frame}
\begin{itemize}
\item Fix $n$ (time) and let $i$ (space) float over all of the interior points of our rod
\begin{itemize}
\item Boundary conditions $u(x_1,t)$ and $u(x_N,t)$ are known for all $t$
\item Interior points
\begin{align*}
u_2^n	&=u_2^{n+1}\Big[1+2r\Big]-r\Big[u_{1}^{n+1}+u_{3}^{n+1}\Big] 	\\
u_3^n	&=u_3^{n+1}\Big[1+2r\Big]-r\Big[u_{2}^{n+1}+u_{4}^{n+1}\Big] 	\\
u_4^n	&=u_4^{n+1}\Big[1+2r\Big]-r\Big[u_{3}^{n+1}+u_{5}^{n+1}\Big] 	\\
.		&	\\
.		&	\\
u_{N-2}^n	&=u_{N-2}^{n+1}\Big[1+2r\Big]-r\Big[u_{N-3}^{n+1}+u_{5}^{N-1}\Big] 	\\
u_{N-1}^n	&=u_{N-1}^{n+1}\Big[1+2r\Big]-r\Big[u_{N-2}^{n+1}+u_{N}^{n+1}\Big] 	\\
\end{align*}
\end{itemize}
\end{itemize}
\end{frame}

\begin{frame}
We move our known terms $u_1^{n+1}$ and $u_N^{n+1}$ to the RHS. 
\vspace{.2in}

Recognizing the pattern we build the linear system:
\begingroup
\fontsize{6pt}{10pt}\selectfont
\[
\left(\begin{array}{c c c c c c c}  1+2r	&-r		&0		&0	&.	&.		&0		\\
										-r		&1+2r	&-r		&0	&0	&.		&0		\\
										0		&-r		&1+2r	&-r	&0	&.		&0		\\
										.		&		&.		&.	&.	&		&.		\\
										.		&		&.		&.	&.	&		&.		\\
										.		&		&.		&.	&.	&		&.		\\
										0		&0		&.		&.	&-r	&1+2r	&-r		\\
										0		&0		&.		&.	&0	&-r		&1+2r	\\ 
\end{array}\right)	
\left(\begin{array}{c} u_2^{n+1}\\u_3^{n+1}\\u_4^{n+1}\\.\\.\\.\\u_{N-2}^{n+1}\\u_{N-1}^{n+1}
\end{array}\right)
=\left(\begin{array}{c} u_2^n+ru_1^{n+1}\\u_3^n\\u_4^n\\.\\.\\.\\u_{N-2}^n\\u_{N-1}^n+ru_N^{n+1}
\end{array}\right)
\]
\endgroup
\end{frame}

\begin{frame}
\[
\begin{array}{c c c}
\includegraphics[scale=.3]{fig1} 	&	\includegraphics[scale=.3]{imp_euler_fig1}		&	\includegraphics[scale=.3]{implicit_fig1}	\\
\text{Euler's} 						&	\text{Improved Euler's}									&	\text{Implicit Time Stepping}  
\end{array}	
\]
\end{frame}


\begin{frame}
\[
\begin{array}{c c}
\includegraphics[scale=.5]{implicit_fig3} 	&	\includegraphics[scale=.5]{implicit_fig4} 
\end{array}	
\]
\end{frame}


\begin{frame}
We can greatly increase the diffusivity constant and the approximation remains stable
\[
\begin{array}{c c}
\includegraphics[scale=.5]{implicit_fig5} 	&	\includegraphics[scale=.5]{implicit_fig6} 
\end{array}	
\]
\end{frame}


\begin{frame}
\[
\begin{array}{c c}
\includegraphics[scale=.5]{implicit_fig7} 	&	\includegraphics[scale=.5]{implicit_fig8} 
\end{array}	
\]
\end{frame}

\begin{frame}
\begin{center}
Exploring Error
\end{center}
\begin{itemize}
\item Simplify IC and Boundary conditions so the we can solve for an explicit solution
\begin{itemize}
\item IC
$$u(x,0)=\sin (\pi x)+0.2\sin (10\pi x)$$
\item Boundary conditions 
$$u(0,t)=u(1,t)=0$$
\item Guess a solution to $u_t=ku_{xx}$ 
$$u(t,x)=e^{-\pi ^2kt}\sin(\pi x)+0.2e^{-(10\pi)^2kt}\sin(10\pi x)$$
\end{itemize}
\end{itemize}
\end{frame}

\begin{frame}
\begin{align*}
u_t 	&=\frac{\partial}{\partial t}\Big[u(x,t)\Big]			\\
		&=	\frac{\partial}{\partial t}\Big[e^{-\pi ^2kt}\sin(\pi x)+0.2e^{-(10\pi)^2kt}\sin(10\pi x)\Big] 	\\
		&=	-\pi ^2ke^{-\pi ^2kt}\sin(\pi x)+0.2\left[-(10\pi)^2k\right]e^{-(10\pi)^2kt}\sin(10\pi x)					\\
		&=	k\Big[-\pi ^2e^{-\pi ^2kt}\sin(\pi x)+0.2\left[-(10\pi)^2k\right]e^{-(10\pi)^2t}\sin(10\pi x)\Big]		\\
		&=k\frac{\partial ^2u}{\partial x^2}\Big[e^{-\pi ^2kt}\sin(\pi x)+0.2e^{-(10\pi)^2kt}\sin(10\pi x)\Big]		\\
		&=k\frac{\partial ^2u}{\partial x^2}\Big[u(x,t)\Big]	\\
		&=ku_{xx}
\end{align*}
\end{frame}

\begin{frame}
Plots with $M=2000, N=200, k=0.10 \rightarrow r=0.2972$
\[
\begin{array}{c c c}
\includegraphics[scale=.25]{exact} 	&	\includegraphics[scale=.25]{euler_error_1}	&	\includegraphics[scale=.25]{euler_error_2}	\\
\text{Exact solution}				&	\text{Euler's method}						&	\text{Error}
\end{array}
\]
At which point in time should we sample the error across the rod?
\begin{itemize}
\item $t$ large and $t\approx 0$ the error is nearly zero
\item Error appears non-trivial at $t\approx 0.10$ seconds
\end{itemize}
\end{frame}

\begin{frame}
Improved Euler's 
\[
\begin{array}{c c c}
\includegraphics[scale=.25]{exact} 	&	\includegraphics[scale=.25]{imp_euler_error_1}	&	\includegraphics[scale=.25]{imp_euler_error_2}	\\
\text{Exact solution}				&	\text{Improved Euler's method}						&	\text{Error}
\end{array}
\]
\end{frame}

\begin{frame}
Implicit time stepping
\[
\begin{array}{c c c}
\includegraphics[scale=.25]{exact} 	&	\includegraphics[scale=.25]{implicit_error_1}	&	\includegraphics[scale=.25]{implicit_error_2}	\\
\text{Exact solution}				&	\text{Implicit time step method}						&	\text{Error}									\\
									&			&	\includegraphics[scale=.25]{implicit_error_3}													
\end{array}
\]
\end{frame}

\begin{frame}
Euler's method at $t=.1$ seconds
\[
\begin{array}{c c c}
\includegraphics[scale=.2]{error_norm_euler4} 	&	\includegraphics[scale=.2]{error_norm_euler5}	&	\includegraphics[scale=.2]{error_norm_euler6}
\end{array}
\]
\end{frame}

\begin{frame}
Improved Euler's method at $t=.1$ seconds
\[
\begin{array}{c c c}
\includegraphics[scale=.2]{error_norm_imp_euler3} 	&	\includegraphics[scale=.2]{error_norm_imp_euler4}	&	\includegraphics[scale=.2]{error_norm_imp_euler5}
\end{array}
\]
\end{frame}



%\item Requires solving the linear system  $$A\vec{u}_n=\vec{u}_{n+1}$$
%\begin{itemize} 
%\item $\vec{u}_n$ represents values at time $n$ 
%\item $\vec{u}_{n+1}$ represents values at time $n+1$ 
%\end{itemize}

%#############################################################################
%#############################################################################
%
%       CONCLUSION
%
\section{Remarks}\label{Remarks}

\begin{frame}\frametitle{Remarks}
\end{frame}

%#############################################################################
%#############################################################################
%
%       ACKNOWLEDGMENTS
%
\section{Acknowledgments}\label{Acknowledgments}

\begin{frame}\frametitle{Acknowledgments}
\end{frame}

%##############################################################################
%##############################################################################
%
%       END OF PRESENTATION
%
\end{document}
